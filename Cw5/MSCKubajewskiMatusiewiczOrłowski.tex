\documentclass{article}
\usepackage{graphicx} % Required for inserting images
\usepackage[utf8]{inputenc}
\usepackage[T1]{fontenc}
\usepackage{karnaugh-map}
\usepackage[polish]{babel}
\usepackage{subcaption}
\usepackage{float}
\usepackage{enumitem}
\usepackage{url}
\usepackage{array}
\usepackage{indentfirst}
\usepackage{amsmath}
\usepackage{pgfplots}
\pgfplotsset{compat=1.17}
\usepackage{longtable}

\begin{document}

\begin{center}
	% Używamy @{} aby usunąć zewnętrzne marginesy
	% Używamy 'm' (z pakietu 'array') dla środkowania w pionie
	
	\begin{tabular}{@{}|m{0.65\textwidth}|m{0.3\textwidth}|@{}}
		\hline
		% --- PIERWSZY WIERSZ LOGICZNY (wg obrazka) ---
		% Komórka 1,1 (Lewa-góra)
		Wydział Informatyki Politechniki Białostockiej \newline
		Przedmiot: Modułowe systemy cyfrowe
		&
		% Komórka 1,2 (Prawa-góra)
		Data: 26.11.2025 \\ \hline
		
		% --- DRUGI WIERSZ LOGICZNY (wg obrazka) ---
		% Komórka 2,1 (Lewa-środek)
		Zajęcia nr 5 \newline
		Temat: Przetworniki A/C i C/A \newline
		
		& % <-- Separator kolumn dla wiersza 2
		
		% Komórka 2,2 (Prawa-środek)
		
		\\
		
		% --- TRZECI WIERSZ LOGICZNY (wg obrazka) ---
		% Komórka 3,1 (Lewa-dół)
		Grupa: Lab 8 \newline
		Imię i nazwisko: \newline
		Kamil Kubajewski, Jakub Matusiewicz, Bartosz Orłowski
		
		& % <-- Separator kolumn dla wiersza 3
		
		% Komórka 3,2 (Prawa-dół)
		Prowadzący: \newline
		dr hab. inż. Sławomir Zieliński \\ \hline
	\end{tabular}
\end{center}
\vspace*{-2em}
\section{Cel ćwiczeń}
Celem ćwiczenia jest zapoznanie sie z budową i zasadą działania przetworników analogowocyfrowych, cyfrowo-analogowych oraz układów kaskadowych.
\vspace*{-1em}
\section{Podstawa teoretyczna}

Na zajęciach zapoznaliśmy się z budową i zasadą działania przetworników analogowo-cyfrowych (A/C) oraz cyfrowo-analogowych (C/A). Układy te stanowią niezbędny element współczesnych systemów pomiarowych i sterujących, umożliwiając komunikację między światem rzeczywistym, w którym sygnały mają charakter analogowy, a systemami cyfrowymi operującymi na wartościach dyskretnych, na przykład komputerami.

Przetwornik (A/C) jest to urządzenie, na którego wejście podajemy napięcie w określonym zakresie, a na wyjściu otrzymujemy liczbę dwójkową, która odpowiada wartości tego napięcia. Im więcej bitów ma liczba wyjściowa, tym dokładniejszy jest dany przetwornik A/C.

Pierwszym etapem jest próbkowanie, które polega na pobieraniu chwilowych wartości napięcia wejściowego w określonych, równych odstępach czasu. 

Drugim etapem przetwarzania A/C jest kwantowanie. Następuje kodowanie, czyli przypisanie cyfrowej wartości każdemu poziomowi. Celem jest uproszczenie sygnału do postaci cyfrowej. To, jak duża jest pula tych dostępnych do wyboru liczb, zależy od liczby bitów przetwornika ($N$) i wynosi dokładnie $2^N$.

W drugiej części zajęć zapoznaliśmy się z procesem odwrotnym, czyli przetwarzaniem cyfrowo-analogowym (C/A). Polega on na zamianie słowa cyfrowego (kodu binarnego) na odpowiadającą mu wartość napięcia wyjściowego.

\section{Przebieg ćwiczeń}

\subsection{Zadanie 1}
Zrealizować przetwornik \textbf{analogowo-cyfrowy (A/C)} wykorzystując moduł laboratoryjny \textbf{DB22}.\\
\begin{figure}[h]
    \centering
    \includegraphics[width=1\textwidth]{obwod1.PNG}
    \caption{Obwód do zadania 1}
    \label{fig:moj_obrazek}
\end{figure}

Zgodnie ze schematem blokowym modułu, układ składa się z komparatora, licznika 4-bitowego oraz przetwornika C/A w pętli sprzężenia zwrotnego. Procedura weryfikacji polegała na podaniu napięcia na wejście analogowe (\textbf{Analog Input}, $V_i$). Następnie wartość napięcia była stopniowo regulowana. Poprawność konwersji weryfikowano, obserwując stan diod LED podłączonych do wyjść cyfrowych przetwornika ($B_0$-$B_3$), które reprezentowały wynikową wartość binarną napięcia wejściowego.
\begin{longtable}{|l|l|l|}
    \caption{Tabela przejściowa} \label{tab:tabela_przejsciowa} \\
    
    % --- Definicja nagłówka na pierwszej stronie ---
    \hline
    V\_in [V] & Słowo\_dec & Słowo \\ \hline
    \endfirsthead
    
    % --- Definicja nagłówka na kolejnych stronach ---
    \multicolumn{3}{c}%
    {{\bfseries \tablename\ \thetable{} -- ciąg dalszy z poprzedniej strony}} \\
    \hline
    V\_in [V] & Słowo\_dec & Słowo \\ \hline
    \endhead
    
    % --- Definicja stopki na dole stron (oprócz ostatniej) ---
    \hline \multicolumn{3}{|r|}{{Ciąg dalszy na następnej stronie}} \\ \hline
    \endfoot
    
    % --- Definicja stopki na ostatniej stronie ---
    \hline
    \endlastfoot

    % --- DANE ---
    0 & 0 & 0000 \\ \hline
    0,1 & 0 & 0000 \\ \hline
    0,2 & 0 & 0000 \\ \hline
    0,3 & 1 & 0001 \\ \hline
    0,4 & 1 & 0001 \\ \hline
    0,5 & 1 & 0001 \\ \hline
    0,6 & 2 & 0010 \\ \hline
    0,7 & 2 & 0010 \\ \hline
    0,8 & 2 & 0010 \\ \hline
    0,9 & 3 & 0011 \\ \hline
    1 & 3 & 0011 \\ \hline
    1,1 & 3 & 0011 \\ \hline
    1,2 & 4 & 0100 \\ \hline
    1,3 & 4 & 0100 \\ \hline
    1,4 & 4 & 0100 \\ \hline
    1,5 & 5 & 0101 \\ \hline
    1,6 & 5 & 0101 \\ \hline
    1,7 & 5 & 0101 \\ \hline
    1,8 & 6 & 0110 \\ \hline
    1,9 & 6 & 0110 \\ \hline
    2 & 6 & 0110 \\ \hline
    2,1 & 7 & 0111 \\ \hline
    2,2 & 7 & 0111 \\ \hline
    2,3 & 7 & 0111 \\ \hline
    2,4 & 7 & 0111 \\ \hline
    2,5 & 8 & 1000 \\ \hline
    2,6 & 8 & 1000 \\ \hline
    2,7 & 9 & 1001 \\ \hline
    2,8 & 9 & 1001 \\ \hline
    2,9 & 9 & 1001 \\ \hline
    3 & 10 & 1010 \\ \hline
    3,1 & 10 & 1010 \\ \hline
    3,2 & 10 & 1010 \\ \hline
    3,3 & 11 & 1011 \\ \hline
    3,4 & 11 & 1011 \\ \hline
    3,5 & 11 & 1011 \\ \hline
    3,6 & 12 & 1100 \\ \hline
    3,7 & 12 & 1100 \\ \hline
    3,8 & 12 & 1100 \\ \hline
    3,9 & 13 & 1101 \\ \hline
    4 & 13 & 1101 \\ \hline
    4,1 & 13 & 1101 \\ \hline
    4,2 & 13 & 1101 \\ \hline
    4,3 & 14 & 1110 \\ \hline
    4,4 & 14 & 1110 \\ \hline
    4,5 & 14 & 1110 \\ \hline
    4,6 & 15 & 1111 \\ \hline
    4,7 & 15 & 1111 \\ \hline
    4,8 & 15 & 1111 \\ \hline
\end{longtable}
\newpage
\begin{figure}[h]
    \centering
    \includegraphics[width=1\textwidth]{wykres1.PNG}
    \caption{Obwód do zadania 1}
    \label{fig:moj_obrazek}
\end{figure}

\newpage
\subsection{Zadanie 2}
Zrealizować przetwornik cyfrowo-analogowy (C/A) wykorzystując moduł laboratoryjny DB16.\\
\begin{figure}[h]
    \centering
    \includegraphics[width=1\textwidth]{obwod2.PNG}
    \caption{Obwód do zadania 2}
    \label{fig:moj_obrazek}
\end{figure}

Procedura pomiarowa polegała na podłączeniu multimetru cyfrowego do wyjścia analogowego układu ($V_o$). Następnie na wejścia cyfrowe ($D_0$ - $D_3$, gdzie $D_3$ to MSB) podawano sekwencyjnie 4-bitowe słowa kodowe, począwszy od wartości $0000_2$ ($0_{10}$), aż do $1111_2$ ($15_{10}$). Dla każdej kombinacji wejściowej zmierzono i zanotowano wartość napięcia wyjściowego, co pozwoliło na wyznaczenie charakterystyki przetwarzania układu.\\
\begin{longtable}{|l|l|l|}
    \caption{Charakterystyka przetwarzania przetwornika C/A (pomiary)} \label{tab:pomiary_dac} \\
    
    % --- Nagłówek na pierwszej stronie ---
    \hline
    Słowo\_dec & Słowo & V\_o [V] \\ \hline
    \endfirsthead
    
    % --- Nagłówek na kolejnych stronach (jeśli tabela pęknie) ---
    \multicolumn{3}{c}%
    {{\bfseries \tablename\ \thetable{} -- ciąg dalszy z poprzedniej strony}} \\
    \hline
    Słowo\_dec & Słowo & V\_o [V] \\ \hline
    \endhead
    
    % --- Stopka na stronach przejściowych ---
    \hline \multicolumn{3}{|r|}{{Ciąg dalszy na następnej stronie}} \\ \hline
    \endfoot
    
    % --- Stopka końcowa ---
    \hline
    \endlastfoot

    % --- DANE ---
    0 & 0000 & 0 \\ \hline
    1 & 0001 & -0,723 \\ \hline
    2 & 0010 & -1,449 \\ \hline
    3 & 0011 & -2,193 \\ \hline
    4 & 0100 & -2,892 \\ \hline
    5 & 0101 & -3,633 \\ \hline
    6 & 0110 & -4,38 \\ \hline
    7 & 0111 & -5,14 \\ \hline
    8 & 1000 & -5,74 \\ \hline
    9 & 1001 & -6,47 \\ \hline
    10 & 1010 & -7,21 \\ \hline
    11 & 1011 & -7,97 \\ \hline
    12 & 1100 & -8,69 \\ \hline
    13 & 1101 & -9,45 \\ \hline
    14 & 1110 & -10,21 \\ \hline
    15 & 1111 & -10,98 \\ \hline
\end{longtable}

\begin{figure}[h]
    \centering
    \includegraphics[width=1\textwidth]{wykres2.PNG}
    \caption{Wykres do zadania 2}
    \label{fig:moj_obrazek}
\end{figure}

\section{Wnioski}
Po przeprowadzeniu ćwiczeń możemy potwierdzić teoretyczne zasady działania przetworników analogowo-cyfrowych i cyfrowo-analogowych. Pomiary potwierdziły, że przetworniki te stanowią kluczowy element łączący systemy analogowe z cyfrowymi.

Podczas korzystania z przetwornika A/C (moduł DB22) mogliśmy zauważyć charakterystyczną, schodkową zmianę wartości wyjściowej. Jest to dowód na działanie procesu kwantowania, gdzie ciągłe napięcie wejściowe jest przypisywane do skończonych wartości cyfrowych. Dla wykorzystanego układu 4-bitowego rozdzielczość pomiarowa wynosiła około $0,3\,V$. Oznacza to, że układ nie reagował na zmiany napięcia mniejsze niż ta wartość, co obrazuje błąd kwantowania.

W drugiej części zajęć, korzystając z przetwornika C/A (moduł DB16), wyznaczyliśmy jego charakterystykę przejściową. Okazała się ona liniowa, co świadczyło o poprawnej pracy układu. Zauważyliśmy, że napięcia wyjściowe miały wartości ujemne, co wynikało z zastosowania w module wzmacniacza w konfiguracji odwracającej. Średnia zmiana napięcia przypadająca na jeden bit wynosiła około $-0,73\,V$.
\newpage

\section{Protokół}
\begin{figure}[h]
    \centering
    \includegraphics[width=0.8\textwidth]{protokol.PNG}
    \caption{Protokół}
    \label{fig:moj_obrazek}
\end{figure}


\end{document}
